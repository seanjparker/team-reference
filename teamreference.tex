\documentclass[a4paper]{article}

\usepackage{color}
\usepackage{listings}

\definecolor{dkgreen}{rgb}{0,0.6,0}
\definecolor{gray}{rgb}{0.5,0.5,0.5}
\definecolor{mauve}{rgb}{0.58,0,0.82}

\lstset{frame=tb,
  language=Java,
  aboveskip=3mm,
  belowskip=3mm,
  showstringspaces=false,
  columns=flexible,
  basicstyle={\small\ttfamily},
  numbers=none,
  numberstyle=\tiny\color{gray},
  keywordstyle=\color{blue},
  commentstyle=\color{dkgreen},
  stringstyle=\color{mauve},
  breaklines=true,
  breakatwhitespace=true,
  tabsize=3
}

\usepackage[english]{babel}
\usepackage[utf8]{inputenc}
\usepackage{fancyhdr}

\pagestyle{fancy}
\fancyhf{}
\rhead{Page \thepage}
\lhead{The University of Manchester - Ivan Donat Pupovac}
\rfoot{Page \thepage}

\title{Team Reference Document}
\author{The University of Manchester - Spirit Monkeys}


\begin{document}

\maketitle
\newpage

\section{Algorithms}


\subsection{Maximum subarray sum}

Consider the subproblem of finding the maximum-sum subarray that ends at position k. There are two possibilities:
\begin{enumerate}
  \item The subarray only contains the element at position k
  \item The subarray consists of a subarray that ends at position k - 1, followed by the element at position k
\end{enumerate}


\begin{lstlisting}
int best = 0, sum = 0;
for (int k = 0; k < n; k++) {
  sum = max(array[k], sum + array[k]);
  best = max(best, sum);
}
\end{lstlisting}

\end{document}
